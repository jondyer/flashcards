\documentclass[12pt]{article}
\special{papersize=3in,5in}

\usepackage{amssymb,amsmath}
\newcommand{\R}{\mathbb{R}}
\newcommand{\N}{\mathbb{N}}

\pagestyle{empty}
\setlength{\parindent}{0in}
\begin{document}

\begin{centering}
  Definition of a metric\\
\end{centering}
If X is a set, then a \textbf{metric} on X is a function $d:X\times X \rightarrow \R$ satisfying:
\begin{itemize}
  \item $d(x,y) = 0 iff x=y$
  \item $d(x,y) = d(y,x)$
  \item $d(x,y) \leq d(x,z) + d(z,y)$
\end{itemize}
A pair (X,d) is called a \textbf{metric space}.\par\medskip

Give three examples of a metric space
\begin{itemize}
  \item $\R^n$ with "usual (Euclidean) metric": $d_e(a,b) = \sqrt{\sum_{i=1}^n (a_i - b_i)}$
  \item $\R^2$ with "Manhattan metric": $d_m(x,y) = |x_1 - y_1| + |x_2 - y_2|$
  \item $\N$ with the "discrete metric": $d_d(n,k) = \begin{cases}
      0, & \text{if}\ n=k \\
      1, & \text{if}\ n \neq k
    \end{cases}$
\end{itemize}



\end{document}
